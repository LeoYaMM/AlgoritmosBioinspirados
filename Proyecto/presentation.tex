\documentclass{beamer}
\usepackage{tikz}
\usepackage{graphicx}
\usepackage{helvet}

\graphicspath{{imagenes/}}

\renewcommand{\familydefault}{\sfdefault}

\title{Celullar Automata}
\author{Castillo Reyes, Diego\\
        Escamilla Reséndiz, Aldo\\
        Muñoz Gonzalez, Eduardo\\
        Yañez Martinez, Marthon Leobardo\\}
\date{\today}

\begin{document}

\begin{frame}[plain]
    \titlepage
    \begin{tikzpicture}[remember picture,overlay]
        \node[at={(current page.south west)}, anchor=south west, xshift=-0.5cm, yshift=0.1cm, opacity=0.2] {\includegraphics[width=0.5\paperwidth]{IPNLogo.png}};
    \end{tikzpicture}
\end{frame}

\section{Introducción}
\begin{frame}
    \frametitle{Introduction}
    Cellular automata (CA) are \textit{discrete, abstract computational systems} that have proved useful both as general models of complexity and as more specific representations of non-linear dynamics in a variety of scientific fields.
    \begin{tikzpicture}[remember picture,overlay]
        \node[at={(current page.south east)}, anchor=south east, xshift=-0.3cm, yshift=2.5cm, opacity=0.2] {\includegraphics[width=0.6\paperwidth]{escomLogo.png}};
    \end{tikzpicture}
\end{frame}

\section{Background}
\begin{frame}
    \frametitle{Background}
    Aquí va el contenido de la metodología.
\end{frame}

\begin{frame}
    \frametitle{Título de la Diapositiva}
    Contenido de la diapositiva.
\end{frame}

\end{document}
