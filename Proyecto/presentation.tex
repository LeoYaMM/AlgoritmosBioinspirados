\documentclass{beamer}
\usepackage{tikz}
\usepackage{graphicx}
\usepackage{helvet}
\usepackage[linesnumbered,ruled,vlined]{algorithm2e}

\graphicspath{{imagenes/}}

\renewcommand{\familydefault}{\sfdefault}

\title{Cellular Automata}
\author{\texorpdfstring{Castillo Reyes, Diego\\ Escamilla Reséndiz, Aldo\\ Muñoz Gonzalez, Eduardo\\ Yañez Martinez, Marthon Leobardo}{Castillo Reyes, Diego, Escamilla Reséndiz, Aldo, Muñoz Gonzalez, Eduardo, Yañez Martinez, Marthon Leobardo}}
\date{\today}

\begin{document}

\begin{frame}[plain]
    \titlepage
    \begin{tikzpicture}[remember picture,overlay]
        \node[at={(current page.south west)}, anchor=south west, xshift=-0.5cm, yshift=0.1cm, opacity=0.2] {\includegraphics[width=0.5\paperwidth]{IPNLogo.png}};
    \end{tikzpicture}
\end{frame}

\section{Introducción}
\begin{frame}
    \frametitle{Introduction}
    Cellular automata (CA) are \textit{discrete, abstract computational systems} that have proved useful both as general models of complexity and as more specific representations of non-linear dynamics in a variety of scientific fields.
    \begin{tikzpicture}[remember picture,overlay]
        \node[at={(current page.south east)}, anchor=south east, xshift=-0.3cm, yshift=2.5cm, opacity=0.2] {\includegraphics[width=0.6\paperwidth]{escomLogo.png}};
    \end{tikzpicture}
\end{frame}

\section{Background}
\begin{frame}
    \frametitle{Background}
    Cellular automata (CA) were conceptualized by Stanislaw Ulam and John Von Neumann in the 1940s at the Los Alamos National Laboratory. Von Neumann's extensive work on self-replicating automata was published posthumously in 1966. A CA consists of a one-dimensional array of cells that evolve over discrete time steps.
\end{frame}

\section{Cellular Automata Algorithm}
\begin{frame}
    \frametitle{Cellular Automata Algorithm}
    \begin{center}
        \scalebox{0.8}{
            \begin{minipage}{1.2\textwidth}
                \begin{algorithm}[H]
                    \caption{Basic Cellular Automaton}
                    \KwIn{\texttt{gridWidth}: Width of the grid, \texttt{gridHeight}: Height of the grid, \texttt{states}: Set of possible states for the cells, \texttt{neighborhood}: Set of relative positions defining the neighborhood of each cell, \texttt{rules}: Set of state transition rules, \texttt{maxTimeSteps}: Maximum number of time steps}
                    \KwOut{The final state of the grid}
                    
                    Initialize \texttt{gridHeight} $\times$ \texttt{gridWidth}, set the initial states on the grid and create \texttt{newGrid} as a copy of the grid.\;
                    
                    \While{$i$ < \texttt{maxTimeSteps}}{
                        \For{$x$ in \texttt{gridWidth}}{
                            \For{$y$ in \texttt{gridHeight}}{
                                \texttt{neighbors} = getNeighbors(\texttt{grid}, \texttt{neighborhood}, $x$, $y$)\;
                                \texttt{newGrid}[$x$][$y$] = applyRules(\texttt{grid}[$x$][$y$], \texttt{neighbors}, \texttt{rules})\;
                            }
                        }
                        Display the state of \texttt{newGrid}\;
                        \texttt{grid} = \texttt{newGrid}\;
                        $i$++\;
                    }
                \end{algorithm}
            \end{minipage}
        }
    \end{center}
\end{frame}

\section{Parameters of Cellular Automata}
\begin{frame}
    \frametitle{Parameters Required on a Cellular Automata (I)}
    Several key parameters determine the structure, behavior, and evolution of a Cellular Automaton (CA):

    \begin{itemize}
        \item \textbf{Grid Structure:}
        \begin{itemize}
            \item \textit{Dimension:} One-dimensional (line of cells) or two-dimensional (grid of cells).
            \item \textit{Size:} Number of cells in each dimension (e.g., 100 cells or 100x100 grid).
        \end{itemize}
        \item \textbf{Cell States:}
        \begin{itemize}
            \item \textit{Number of States:} Possible states each cell can be in (e.g., 0 or 1).
            \item \textit{Initial Configuration:} Initial state of cells at time \( t = 0 \).
        \end{itemize}
        \item \textbf{Neighborhood:}
        \begin{itemize}
            \item \textit{Radius:} Distance defining the neighborhood of a cell.
            \item \textit{Shape:} Common shapes include von Neumann and Moore neighborhoods.
        \end{itemize}
    \end{itemize}      
\end{frame}

\begin{frame}
    \frametitle{Parameters Required on a Cellular Automata (II)}
    \begin{itemize}
        \item \textbf{Transition Function:}
        \begin{itemize}
            \item \textit{Local Rule:} Determines the next state of a cell based on its current state and neighbors.
            \item \textit{Update Scheme:} Synchronous (all cells) or asynchronous (one cell at a time).
        \end{itemize}
        \item \textbf{Boundary Conditions:}
        \begin{itemize}
            \item \textit{Periodic:} Cells wrap around to the opposite edge.
            \item \textit{Fixed:} Edge cells have a fixed state.
            \item \textit{Reflective:} Edge cells influenced by reflecting states within the grid.
        \end{itemize}
        \item \textbf{Time Steps:}
        \begin{itemize}
            \item \textit{Discrete Time:} Evolution occurs in discrete time steps.
            \item \textit{Duration:} Number of time steps the CA runs.
        \end{itemize}
        \item \textbf{Output and Visualization:}
        \begin{itemize}
            \item \textit{State Representation:} Visual or numerical representation of cell states.
            \item \textit{Data Collection:} Recording states of cells for analysis and visualization.
        \end{itemize}
    \end{itemize}
\end{frame}

\section{Versions of Cellular Automata}
\begin{frame}
    \frametitle{Versions of Cellular Automata (I)}
    CA have evolved into various versions based on their dimensionality, state complexity, neighborhood structure, and rule sets.

    \textbf{By Dimensionality:}
    \begin{itemize}
        \item \textit{One-Dimensional:} Cells have two neighbors (left and right). Example: Elementary cellular automata by Stephen Wolfram.
        \item \textit{Two-Dimensional:} Cells have multiple neighbors. Example: Conway’s Game of Life.
        \item \textit{Higher-Dimensional:} Three-dimensional and higher, used for complex simulations.
    \end{itemize}
    \textbf{By Cell States:}
    \begin{itemize}
        \item \textit{Binary:} Cells in two states (0 or 1).
        \item \textit{Multi-State:} Cells in more than two states.
        \item \textit{Continuous-State:} Cells take on a range of continuous values.
    \end{itemize}
\end{frame}

\begin{frame}
    \frametitle{Versions of Cellular Automata (II)}
    \textbf{By Neighborhood Type:}
    \begin{itemize}
        \item \textit{Von Neumann:} Four orthogonal neighbors (N, S, E, W).
        \item \textit{Moore:} Eight surrounding cells (orthogonal and diagonal).
        \item \textit{Extended:} Larger neighborhoods including more distant neighbors.
    \end{itemize}
    \textbf{By Rule Type:}
    \begin{itemize}
        \item \textit{Deterministic:} Next state determined by current state and neighbors.
        \item \textit{Probabilistic:} Next state determined probabilistically.
        \item \textit{Totalistic:} State depends on the total number of particular states in the neighborhood.
    \end{itemize}
    \textbf{By Boundary Conditions:}
    \begin{itemize}
        \item \textit{Periodic:} Grid wraps around, creating a toroidal structure.
        \item \textit{Fixed:} States at the boundaries are fixed.
        \item \textit{Reflective:} States at boundaries reflect the states of their neighbors inside the grid.
    \end{itemize}
\end{frame}

\begin{frame}
    \frametitle{Special Types of Cellular Automata}
    \textbf{Special Types:}
    \begin{itemize}
        \item \textit{Elementary CA:} One-dimensional binary CA by Stephen Wolfram, 256 possible rules.
        \item \textit{Conway’s Game of Life:} Two-dimensional binary CA by John Conway.
        \item \textit{Langton’s Ant:} Two-dimensional Turing machine with simple rules and complex behavior.
        \item \textit{Fuzzy CA:} Combines CA with fuzzy logic for uncertainty in state transitions.
        \item \textit{Quantum CA:} Extends CA principles to quantum computing, allowing superpositions of states.
    \end{itemize}
    These versions enable modeling and simulation from simple theoretical constructs to complex real-world phenomena.
\end{frame}

\section{Analogy with Nature}
\begin{frame}
    \frametitle{Analogy with Nature (I)}
    CA come in various forms, inspired by different aspects of natural systems. These versions differ based on their dimensionality, complexity, neighborhood structure, and rules.
    
    \textbf{Dimensionality:}
    \begin{itemize}
        \item \textit{One-Dimensional:} Simple, theoretical studies (e.g., line of ants).
        \item \textit{Two-Dimensional:} Grid pattern, interactions with neighbors (e.g., moss on a rock).
        \item \textit{Higher-Dimensional:} Complex models (e.g., 3D growth of crystals).
    \end{itemize}
    
    \textbf{Cell States:}
    \begin{itemize}
        \item \textit{Binary:} Two states, like alive or dead (0 or 1).
        \item \textit{Multi-State:} Intermediate states, similar to leaf growth stages.
        \item \textit{Continuous-State:} Spectrum of states, akin to shades of green in a forest.
    \end{itemize}
\end{frame}

\begin{frame}
    \frametitle{Analogy with Nature (II)}
    \textbf{Neighborhoods:}
    \begin{itemize}
        \item \textit{Von Neumann:} Four orthogonal neighbors (N, S, E, W).
        \item \textit{Moore:} Includes diagonal neighbors (3x3 grid).
    \end{itemize}
    
    \textbf{Rule Types:}
    \begin{itemize}
        \item \textit{Deterministic:} Fixed rules, like seasonal changes.
        \item \textit{Probabilistic:} Randomness, similar to weather patterns.
        \item \textit{Totalistic:} Sum of states in neighborhood, like forest density.
    \end{itemize}
    
    \textbf{Boundary Conditions:}
    \begin{itemize}
        \item \textit{Periodic:} Wrap around, toroidal structure.
        \item \textit{Fixed:} Rigid edge, like shorelines.
        \item \textit{Reflective:} Mimic natural barriers, reflecting influence.
    \end{itemize}
\end{frame}

\begin{frame}
    \frametitle{Special Types of Cellular Automata}
    \textbf{Special Types:}
    \begin{itemize}
        \item \textit{Elementary CA:} Simple rules leading to complex behavior.
        \item \textit{Conway’s Game of Life:} Birth, survival, and death rules creating lifelike patterns.
        \item \textit{Langton’s Ant:} Simple rules resulting in organized paths.
        \item \textit{Fuzzy CA:} Combines CA with fuzzy logic for uncertainty.
        \item \textit{Quantum CA:} Cells in superpositions of states, like quantum particles.
    \end{itemize}
    These versions reflect the richness and variety found in natural systems, allowing for modeling and simulation of phenomena across different fields.
\end{frame}

\section{Usage Examples}
\begin{frame}
    \frametitle{Usage Examples}
    Implementations of Cellular Automata:
    
    \textbf{7.1. A Computational Tumor Growth Model Experience}
    \begin{itemize}
        \item \textit{Overview of Cellular Automata and CNN Integration}
        \item \textit{Implementation Details}
        \item \textit{Results and Validation}
    \end{itemize}
    
    \textbf{7.2. Implementing Fuzzy Cellular Automata in Breast Cancer Image Segmentation}
    \begin{itemize}
        \item \textit{Overview of Cellular Automata and Fuzzy Logic Integration}
        \item \textit{Implementation Details}
        \item \textit{Advantages of the Approach}
        \item \textit{Experimental Results}
    \end{itemize}
\end{frame}

\end{document}
