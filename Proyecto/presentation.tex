\documentclass{beamer}
\usepackage{tikz}
\usepackage{graphicx}
\usepackage{helvet}
\usepackage[linesnumbered,ruled,vlined]{algorithm2e}

\graphicspath{{imagenes/}}

\renewcommand{\familydefault}{\sfdefault}

\title{Cellular Automata}
\author{Castillo Reyes, Diego\\
        Escamilla Reséndiz, Aldo\\
        Muñoz González, Eduardo\\
        Yañez Martinez, Marthon Leobardo\\}
\date{\today}

\begin{document}

\begin{frame}[plain]
    \titlepage
    \begin{tikzpicture}[remember picture,overlay]
        \node[at={(current page.south west)}, anchor=south west, xshift=-0.5cm, yshift=0.1cm, opacity=0.2] {\includegraphics[width=0.5\paperwidth]{IPNLogo.png}};
    \end{tikzpicture}
\end{frame}

\section{Introducción}
\begin{frame}
    \frametitle{Introduction}
    Cellular automata (CA) are \textit{discrete, abstract computational systems} that have proved useful both as general models of complexity and as more specific representations of non-linear dynamics in a variety of scientific fields.
    \begin{tikzpicture}[remember picture,overlay]
        \node[at={(current page.south east)}, anchor=south east, xshift=-0.3cm, yshift=2.5cm, opacity=0.2] {\includegraphics[width=0.6\paperwidth]{escomLogo.png}};
    \end{tikzpicture}
\end{frame}

\section{Background}
\begin{frame}
    \frametitle{Background}
    Cellular automata (CA) were conceptualized by Stanislaw Ulam and John Von Neumann in the 1940s at the Los Alamos National Laboratory. Von Neumann's extensive work on self-replicating automata was published posthumously in 1966. A CA consists of a one-dimensional array of cells that evolve over discrete time steps.
\end{frame}

\section{Cellular Automata Algorithm}
\begin{frame}
    \frametitle{Cellular Automata Algorithm}
    \begin{center}
        \scalebox{0.8}{
            \begin{minipage}{1.2\textwidth}
                \begin{algorithm}[H]
                    \caption{Basic Cellular Automaton}
                    \KwIn{\texttt{gridWidth}: Width of the grid, \texttt{gridHeight}: Height of the grid, \texttt{states}: Set of possible states for the cells, \texttt{neighborhood}: Set of relative positions defining the neighborhood of each cell, \texttt{rules}: Set of state transition rules, \texttt{maxTimeSteps}: Maximum number of time steps}
                    \KwOut{The final state of the grid}
                    
                    Initialize \texttt{gridHeight} $\times$ \texttt{gridWidth}, set the initial states on the grid and create \texttt{newGrid} as a copy of the grid.\;
                    
                    \While{$i$ < \texttt{maxTimeSteps}}{
                        \For{$x$ in \texttt{gridWidth}}{
                            \For{$y$ in \texttt{gridHeight}}{
                                \texttt{neighbors} = getNeighbors(\texttt{grid}, \texttt{neighborhood}, $x$, $y$)\;
                                \texttt{newGrid}[$x$][$y$] = applyRules(\texttt{grid}[$x$][$y$], \texttt{neighbors}, \texttt{rules})\;
                            }
                        }
                        Display the state of \texttt{newGrid}\;
                        \texttt{grid} = \texttt{newGrid}\;
                        $i$++\;
                    }
                \end{algorithm}
            \end{minipage}
        }
    \end{center}
\end{frame}

\section{Parameters of the algorithm}
\begin{frame}
    \frametitle{Parameters of the algorithm}
    \begin{itemize}
        \item \texttt{gridWidth}: Width of the grid.
        \item \texttt{gridHeight}: Height of the grid.
        \item \texttt{states}: Set of possible states for the cells.
        \item \texttt{neighborhood}: Set of relative positions defining the neighborhood of each cell.
        \item \texttt{rules}: Set of state transition rules.
        \item \texttt{maxTimeSteps}: Maximum number of time steps.
    \end{itemize}
\end{frame}

\section{Analogy with nature}
\begin{frame}
    \frametitle{Analogy with nature}
    Cellular automata are used to model complex systems in nature, such as the growth of plants, the spread of diseases, and the behavior of animals. They are also used to model the behavior of crowds, traffic, and other social systems.
\end{frame}

\section{Applications}
\begin{frame}
    \frametitle{Applications}
    Cellular automata have been used in a wide range of applications, these are the most relevant we found:
    \begin{itemize}
        \item A computational tumor growth model experience based on
        molecular dynamics point of view using deep cellular automata.
        \item Implementing Fuzzy Cellular Automata in Breast Cancer
        Image Segmentation.
    \end{itemize}
\end{frame}

\end{document}
