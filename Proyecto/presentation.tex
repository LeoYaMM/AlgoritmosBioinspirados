\documentclass{beamer}

\title{Título de la Presentación}
\author{Tu Nombre}
\date{\today}

\begin{document}

\frame{\titlepage}

\section{Introducción}
\begin{frame}
    \frametitle{Introducción}
    Aquí va el contenido de la introducción.
\end{frame}

\section{Metodología}
\begin{frame}
    \frametitle{Metodología}
    Aquí va el contenido de la metodología.
\end{frame}

\begin{frame}
    \frametitle{Título de la Diapositiva}
    Contenido de la diapositiva.
\end{frame}

\end{document}
