%%%%%%%%%%%%%%%%%%%%%%%%%%%%%%%%%%%%%%%%%%%%%%%%%%%%%%%%%%%
% --------------------------------------------------------
% Tau
% LaTeX Template
% Version 2.4.1 (22/05/2024)
%
% Author: 
% Guillermo Jimenez (memo.notess1@gmail.com)
% 
% License:
% Creative Commons CC BY 4.0
% --------------------------------------------------------
%%%%%%%%%%%%%%%%%%%%%%%%%%%%%%%%%%%%%%%%%%%%%%%%%%%%%%%%%%%
\documentclass[9pt,a4paper,twoside]{tau-class/tau}

%----------------------------------------------------------
% TITLE
%----------------------------------------------------------

\usepackage{enumitem}
\newcount\NewCount

\journalname{ESCOM}
\title{Problema de la Coloración Mínima}%! Modificar con el título del proyecto


%----------------------------------------------------------
% AUTHORS, AFFILIATIONS AND PROFESSOR
%----------------------------------------------------------

\author[a]{Diego Castillo Reyes}
\author[a]{Marthon Leobardo Yañez Martinez}
\author[a]{Aldo Escamilla Resendiz}
\author[a]{Muñoz González Eduardo}

%----------------------------------------------------------

\affil[a]{Investigadores en formación, ESCOM, IPN}


\professor{Dra. Miriam Pescador Rojas}

%----------------------------------------------------------
% FOOTER INFORMATION
%----------------------------------------------------------

\institution{Escuela Superior de Cómputo, IPN}
\footinfo{Problema de la Coloración Mínima}%! Modificar con el título del proyecto
\theday{Junio 25, 2024}
\course{Algoritmos Genéticos}

%----------------------------------------------------------
% ABSTRACT AND KEYWORDS
%----------------------------------------------------------

\begin{abstract}    
    En este trabajo se presenta el problema de la coloración mínima, el cual es un problema NP-completo. 
    Se propone una solución basada en algoritmos genéticos para encontrar la coloración mínima de un grafo.


\end{abstract}

%----------------------------------------------------------

\keywords{Coloración Mínima, Genéticos, Algoritmos, Grafos}%! Modificar con las palabras clave

%----------------------------------------------------------

\begin{document}	
    \maketitle 
    \thispagestyle{firststyle} 
    \tauabstract
    \tableofcontents
%----------------------------------------------------------

\section{Introducción}

    El problema de la coloración mínima es un problema NP-completo, el cual consiste en asignar un color a cada vértice de un grafo de tal manera que dos vértices adyacentes no tengan el mismo color.
    El objetivo es encontrar la coloración mínima, es decir, la menor cantidad de colores posibles para colorear el grafo.
    Este problema es de gran importancia en la teoría de grafos, ya que tiene aplicaciones en la asignación de horarios, asignación de frecuencias, asignación de canales, entre otros.
    En este trabajo se propone una solución basada en PSO, un algortmo de optimización basado en la inteligencia de enjambre, para encontrar la coloración mínima de un grafo.

\section{Antecedentes}

    En una pequeña ciudad de Rusia, K\"onigsberg (Actualmente Kaliningrado, Rusia), existen siete puentes que conectan cuatro islas. Un matemático llamado Leonhard Euler (1707-1783) 
    se preguntó si era posible recorrer todos los puentes una sola vez y regresar al punto de partida. 
    Euler demostró en 1736 que no era posible en su publicación \textit{Solutio problematis ad geometriam situs pertinentis} \cite{euler1736}, y 
    para ello utilizó un grafo para representar las islas y los puentes como se ve en la Figura~\ref*{fig:konigsberg}. 
    \begin{figure}[H]
        \centering
        \includegraphics[width=0.5\columnwidth]{Images/Proy/Euler.png }
        \caption{Grafo de K\"onigsberg}
        \label{fig:konigsberg}
    \end{figure}
    Él demostró utilizando cada puente como una arista y cada isla como un vértice, que no era posible recorrer todos los puentes una sola vez y regresar al punto de partida.
    Siendo este el primer problema de teoría de grafos, el cual es un área de las matemáticas que estudia las relaciones entre los vértices y las aristas de un grafo.  
    Un grafo es un conjunto de vértices y aristas, donde las aristas son pares no ordenados de vértices.
    Posteriormente, en 1852, Francis Guthrie planteó el problema de los cuatro colores, el cual consiste en colorear un mapa de tal manera que dos regiones adyacentes no tengan el mismo color.
    Este problema fue resuelto en 1976 por Kenneth Appel y Wolfgang Haken utilizando una computadora, demostrando que cuatro colores son suficientes para colorear cualquier mapa~\cite{appel1976}.
    A partir de este problema, se han planteado diversos problemas de coloración, entre ellos el problema de la coloración mínima, el cual es un problema NP-completo.
    
\section{Metodología}


\section{Resultados}


\section{Conclusiones}

%%%%%%%%%%%%%%%%%%%%%%%%%%%%%%%%%%%%%%%%%%%%%%%%%%%%%%%%%%%
%%%%%%%%%%%%%%%%%%%%%%%%%%%%%%%%%%%%%%%%%%%%%%%%%%%%%%%%%%%
%%%%%%%%%%%%%%%%%%%%%%%%%%%%%%%%%%%%%%%%%%%%%%%%%%%%%%%%%%%

\printbibliography

\end{document}