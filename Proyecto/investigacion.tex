\documentclass{article}
\usepackage[utf8]{inputenc}
\usepackage[english]{babel}
\usepackage[margin=2cm]{geometry}
\usepackage{graphicx}
\usepackage{float}
\usepackage{titlesec}
\usepackage{caption}
\usepackage{listings}
\usepackage{xcolor}
\usepackage{array}
\usepackage{booktabs}
\usepackage{tabularx}
\usepackage{multicol}
\usepackage{multirow}
\usepackage{amsmath}
\usepackage{hyperref}

\definecolor{codegreen}{rgb}{0,0.6,0}
\definecolor{codegray}{rgb}{0.5,0.5,0.5}
\definecolor{codepurple}{rgb}{0.58,0,0.82}
\definecolor{backcolor}{rgb}{0.95,0.95,0.95}


\lstset{
    basicstyle=\ttfamily,
    inputencoding=utf8,
    extendedchars=true,
    literate=%
    {á}{{\'a}}1
    {é}{{\'e}}1
    {í}{{\'i}}1
    {ó}{{\'o}}1
    {ú}{{\'u}}1
    {ñ}{{\~n}}1
    {Á}{{\'A}}1
    {É}{{\'E}}1
    {Í}{{\'I}}1
    {Ó}{{\'O}}1
    {Ú}{{\'U}}1
    {Ñ}{{\~N}}1
}


\lstdefinestyle{mystyle}{
    backgroundcolor=\color{backcolor},
    commentstyle=\color{codegreen},
    keywordstyle=\color{red},
    numberstyle=\tiny\color{codegray},
    stringstyle=\color{codepurple},
    basicstyle=\ttfamily\footnotesize,
    breakatwhitespace=false,
    breaklines=true,
    captionpos=b,
    keepspaces=true,
    numbers=left,
    showspaces=false,
    showstringspaces=false,
    showtabs=false,
    tabsize=2  
}

\titleformat{\section}
{\huge\bfseries}{\thesection.}{1em}{}
\titleformat{\subsection}
{\large\bfseries}{\thesubsection}{1em}{}

\renewcommand\thesection{\arabic{section}}

\title{\Huge{\textbf{}}\\
\Large{\textbf{Bioinspirated Algorithms\\ Cellular Automaton for Minimum Coloring Graph Problem}}}
\author{Diego Castillo Reyes\\Marthon Leobardo Yañez Martinez\\Aldo Escamilla Resendiz\\Muñoz González Eduardo}

\graphicspath{{imagenes/}}
\begin{document}
    \maketitle
    \newpage
    \tableofcontents
    \newpage
    \section{Introduction}
        In this investigation we will explore the viability of the cellular automaton for the minimum coloring graph problem.
        As the first step we will define what is a cellular automaton and in what consists the problem.\\ 
        Then we will explore the background of each concept, why it was created, and what this technique proposes. After that, 
        we will present the algorithm, which parameters does the algorithm requires. 
        \\Finally we will present examples of the usage of this
        algorithm, and conclude the viability of using cellular automaton for an implementation as the next step in this investigation.
    \newpage

    \begin{multicols}{2}
        \section{Cellular Automaton}
        A cellular automaton is a discrete model studied in computational theory, mathematics,
        physics, complexity science, and theoretical biology. It consists of a grid of cells,
        each of which can be in one of a finite number of states.
        \subsection{Background}


    \end{multicols}
\end{document}